\documentclass{beamer}
\usepackage{pifont}
\newcommand{\xmark}{\ding{55}}
\usetheme{Boadilla}

\title{Disc 2d}
\author{Andy}
\institute{UC Berkeley}
\date{\today}

\begin{document}

\begin{frame}
    \titlepage
\end{frame}
\begin{frame}
    \frametitle{Mini Lecture: One Time Pad}

    \begin{enumerate}[<+->]
        \item Unbreakable encryption method
        \item "a" + "a" = "b", "a" + "c" = "d"
        \item Alice and Bob agree on a random cipher as long as the message 
        \item "addekficjladkfjghe"
        \item "Plain Text" + cipher = encrypted text
        \item "abc" what does this decrypt to?
        \item "dog", "cat", "bob", "lol",.... any 3 letter word...
        \item Discovered by a Civil War Vet / Stanford Trustee, AT+T, some Russians, ... etc.
    \end{enumerate}

\end{frame}

\begin{frame}
    \frametitle{Mini Lecture: RSA}

    \begin{enumerate}[<+->]
        \item James Ellis, Malcolm Williamson of GCHQ discover Diffie-Hellman key exchange, 1969
        \item This method allows 2 parties to agree on some number $n$ without revealing it
        \item Clifford Cocks, of GCHQ, discovers RSA, 1973
        \item Diffie and Hellman rediscover DH key exchange, 1976
        \item Diffie and Hellman can't figure out a good 1 way function.
        \item Ron Rivest, Adi Shamir, and Leonard Adleman rediscover RSA, 1977
        \item Rivest comes up with a good one way function after getting drunk at Passover
        \item Those GCHQ guys eventually get credit for their work in 1997.
        \item RSA is the most copied algorithm in the world!
    \end{enumerate}

\end{frame}

\begin{frame}
    \frametitle{Mini Lecture: RSA}

    \begin{enumerate}[<+->]
        \item Analogy:
        \item Combination locks: Anyone can lock them, only the owner can unlock them
        \item You want to send Andy a message. You ask Andy for his combo lock and a box
        \item You stuff your message in the box, and slam the lock shut
        \item Only Andy knows the combination, so only he can open it.
    \end{enumerate}

\end{frame}

\begin{frame}
    \frametitle{Mini Lecture: RSA Scheme}

    \begin{enumerate}[<+->]
        \item $m$ is message, $N, e$ are 2 public keys, $d$ is secret private key
        \item Bob wants to send Alice a secret message
        \item Alice will generate $N, e, d$, and share $N, e$ with Bob.
        \item Bob will calculate $m^{e}$ mod $N$ and send the result to Alice
        \item Call the encrypted message $c$, for ciphertext
        \item Alice will do $c^{d} \equiv m^{e^{d}} \equiv m^{ed} \equiv m$ mod $N$.
    \end{enumerate}

\end{frame}

\begin{frame}
    \frametitle{Mini Lecture: Totient}

    \begin{enumerate}[<+->]
        \item $\phi(N)$ = number of positive integers relatively prime to $n$, up to $n$.
        \item $\phi(9) = 6$, since $1, 2, 4, 5, 7, 8$
        \item Totient is hard to calculate... unless you know the prime factors!
        \item $\phi(N) = (p-1)(q-1)(r-1)(s-1)...$ where $p,q,r,s, ...$ are prime factors
    \end{enumerate}

\end{frame}

\begin{frame}
    \frametitle{Mini Lecture: RSA Generation}

    \begin{enumerate}[<+->]
        \item Alice will generate $N, e, d$, and share $N, e$ with Bob.
        \item $p, q$ = large prime numbers
        \item $N = pq$
        \item Calclate $\phi(N) = (p - 1)(q -1)$
        \item Choose $e$ st it is relatively prime to $\phi(N)$
        \item $d \equiv e^{-1}$ mod $\phi(N)$
        \item \alert{RSA is secure!!!!!!}
        \item Lets try to break it. We try to solve for $d$. Guess and check? \xmark
        \item We have $e$. We need $\phi(N)$, What is $\phi$. Guess and check? \xmark
        \item If we could factor $N$ into $p,q$, we can get $\phi$. 
        \item Factoring is guessing and checking... \xmark
    \end{enumerate}

\end{frame}

\begin{frame}
    \frametitle{Q3}

    \begin{enumerate}[<+->]
        \item $d = e^{-1}$ mod $(p-1)(q-1)(r-1)$. 
        \item $x^{ed} - x \equiv 0 mod N$
        \item $x(x^{ed - 1} - 1)$ must be divisible by $p, q, r$
        \item $x(x^{k(p-1)(q-1)(r-1) + 1 - 1} - 1)$ must be divisible by $p, q, r$
        \item $x(x^{k(p-1)(q-1)(r-1)} - 1)$ must be divisible by $p, q, r$
        \item Lets show that the above is divisible by $p$.
        \item 2 cases. If $x$ is divisible by $p$, we are done.
        \item If not, then use Fermat's Little Theorum on the right term with mod $p$
        \item $(x^{p-1})^{k(q-1)(r-1)} - 1 \equiv 1 - 1 \equiv 0$ mod $p$.
        \item Thus the eqn is divisble by $p$. Do the same for $q, r$.
        \item \alert{Fact:} $x^{\phi{N}} \equiv 1$ mod $N$
    \end{enumerate}

\end{frame}

\end{document}
