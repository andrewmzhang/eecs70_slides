\documentclass{beamer}
\usetheme{Berkeley}

\title{Disc 2a}
\author{Andy}
\institute{UC Berkeley}
\date{\today}

\begin{document}

\begin{frame}
    \titlepage
\end{frame}

\begin{frame}
    \frametitle{Revist: Disc 1d, Probem 1. Well Ordering Principle}

    \begin{enumerate}[<+->]
        \item WOP says $\exists x : \forall y \in S : x \leq y$
        \item In English: There exists a smallest element in your set (of Naturals).
        \item $(\exists s \in S, 0 \leq s \leq z) \implies \exists x : \forall y \in S x \leq y$
        \item In English: If you set contains an element between $[0, z]$, then your set has a smallest
        element
        \item Base: If you set contains 0, your set has a smallest element. This is true because 0 is by
        definition the smallest natural, so your set contains the smallest possible element.
        \item If your set has a 0 or 1, your set has a smallest element.
        \item If your set has a 0, 1, or 2, your set has a smallest element.
        \item If your set as a number between [0, 100], ...
        \item If your set has a number between $[0, \propto)$, ...
    \end{enumerate}

\end{frame}


\begin{frame}
    \frametitle{Revisit: Disc 1d, Problem 1. WOP (cont.)}
    \begin{enumerate}[<+->]
        \item If your set has a number between $[0, \propto)$, your set has a smallest element.
        \item If your set is non-empty, your set has a smallest element.
    \end{enumerate}
\end{frame}

\begin{frame}
    \frametitle{Revisit: Disc 1f, Problem 1. WOP (ignore.)}
    \begin{enumerate}[<+->]
        \item Inductive Hypo: $(\exists s \in S, 0 \leq s \leq z) \implies \exists x : \forall y \in S x \leq y$
        \item Inductive Step: Lets prove $(\exists s \in S, 0 \leq s \leq z+1) \implies WOP$.
        \item Case 1. The smallest element in $S$ is $z+1$. Then $z+1$ is the minimum element.
        \item Case 2. The smallest element is less than $z+1$, or between $[0, z]$.
        Then we know a smallest element in $S$ exists by inductive hypothesis.
        \item What should you change so that the proof works by simple induction (as opposed to strong
induction)?
        \item For all sets $S$ that contain a number $z'$ such that $z' \leq z$, then $S$ contains
        a smallest element.

    \end{enumerate}
\end{frame}

\begin{frame}
    \frametitle{Problem 1a}
    \begin{enumerate}[<+->]
        \item Problem: Consider a tree with $n \geq 3$ vertices.
        What is the largest possible number of leaves the tree could have?
        Prove that this maximum $m$ is possible to achieve, and further that there cannot
        exist a tree with more than $m$ leaves.
        \item Hint: Why can't a tree have $n$ leaves? Pretend there does exist a tree
        with $n$ leaves. What is each vertice's degree? What is the definition of a tree?
        \item Hint: A tree, by defintion, has a unique, finite length path between 2 vertices.
    \end{enumerate}
\end{frame}

\begin{frame}
    \frametitle{Problem 1a soln}
    \begin{enumerate}[<+->]
        \item Problem: Consider a tree with $n \leq 3$ vertices.
        What is the largest possible number of leaves the tree could have?
        Prove that this maximum $m$ is possible to achieve, and further that there cannot
        exist a tree with more than $m$ leaves.
        \item We know we can construct a tree with $n \geq 3$ nodes and make it have
        $n - 1$ leaves. Do a star graph!
        \item We can't have a tree with $n$ leaves.
        \item Assume such a tree with $n$ leaves exists.
        \item Each vertice has $1$ edge, since each vertex has degree $1$.
        \item Choose an arbitrary $x$ vertex, it has a neighbor, $y$.
        \item $y$ also has only $1$ edge, and it connects to $x$.
        \item The there cannot be a path to $z$, a third arbitrary vertex.
        \item Defintion of tree has been violated!
    \end{enumerate}
\end{frame}

\begin{frame}
    \frametitle{Problem 1b}
    \begin{enumerate}[<+->]
        \item Problem: Prove that every tree with $n \geq 2$ has at least 2 leaves.
        \item Hint: A tree, by defintion, has a unique, finite length path between 2 vertices. We can
        assume (by WOP) that there is a longest path.
    \end{enumerate}
\end{frame}

\begin{frame}
    \frametitle{Problem 1b soln}
    \begin{enumerate}[<+->]
        \item Problem: Prove that every tree with $n \geq 2$ has at least 2 leaves.
        \item Assume the longest path is from x to y (if there are many longest paths, choose one).
        \item x and y are leaves.
        \item Assume that x is not a leaf. It must have degree at least 2. Thus it has a neighbor (we'll call z).
        \item z must not appear in the path from x to y. Why?
        \item We have a cycle. x to a to ... to z to x makes a cycle.
        \item Thus x must be a leaf!
    \end{enumerate}
\end{frame}

\begin{frame}
    \frametitle{Problem 2}
    \begin{enumerate}[<+->]
        \item a. Draw a cube and its dual in a planar graph form.
        \item b. Draw a spanning tree for a cube in planar graph form. Do the same for the dual.
    \end{enumerate}
\end{frame}

\begin{frame}
    \frametitle{Problem 3}
    \begin{enumerate}[<+->]
        \item Hint: Try to prove this for a graph on 7 vertices.
        \item Hint: Where are there definetly not any edges?
    \end{enumerate}
\end{frame}

\begin{frame}
    \frametitle{Problem 4}
    \begin{enumerate}[<+->]
        \item a. Hint: What size is (VxV)?
        \item b. Hint: A planar graph has at most 3v-6 edges.
    \end{enumerate}
\end{frame}

\end{document}
