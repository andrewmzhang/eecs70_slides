\documentclass{beamer}
\usetheme{Berkeley}

\title{Disc 3a}
\author{Andy}
\institute{UC Berkeley}
\date{\today}

\begin{document}

\begin{frame}
    \titlepage
\end{frame}

\begin{frame}
    \frametitle{Implications of RSA}
    
    \begin{enumerate}[<+->]
        \item 2 people can communicate securely despite eavesdroppers!
        \item Eavesdropper CANNOT decipher the conversation
        \item RSA is hard because factoring is ... hard?
        \item \alert{No one has proven factoring is hard}
        \item If you figured out factoring was hard, what would you do?
        \item Keep it secret to yourself? But other people might be mean
        \item Tell the US govt. Get taken out by the CIA.
        \item Tell the world, crash online banking, online security, in minutes.
        \item Its a cool hypothetical
        \item Do let me know if you figure out factoring though!
    \end{enumerate}

\end{frame}


\begin{frame}
    \frametitle{Minilecture: Interpolation}


    \begin{enumerate}[<+->]
        \item I want a function that's
        \item $f(4) = 5$, $f(7) = 10$, $f(2) = 5$.
        \item $\Delta_4(4) = 1$, $d(i) = 0 \forall i, i \neq 4$
        \item $\Delta_7(7) = 1$, $d(i) = 0 \forall i, i \neq 7$
        \item $\Delta_2(2) = 1$, $d(i) = 0 \forall i, i \neq 2$
        \item How does this help?
        \item $f(x) = 5*\Delta_4(x) + 10*\Delta_7(x) + 5 * \Delta_2(x)$
        \item Sanity Check: $f(4) = 5 = 5 * \Delta_4(4) + 10 * \Delta_7(4) + 5 * \Delta_2(4)$
        \item Sanity Check: $f(4) = 5 = 5 * 1 + 10 * 0 + 5 * 0$
        \item $\Delta_i(x)$ is called a Delta function
    \end{enumerate}
\end{frame}


\begin{frame}

    \frametitle{1. Count and Prove}

    \begin{enumerate}[<+->]
        \item CRT says a solution exists within $[0, 105)$
        \item I need something that is 1 mod 3, 0 mod 5, 0 mod 7
        \item $y_1 = (5  * 7) * ((5 * 7)^{-1} mod 3) = 35 * 2 = 70$.
        \item Check: $35 \equiv 1$ mod 3. $35 \equiv 0$ mod 5. $35 \equiv 0$ mod 7.  
        \item $y_2 = (3*7) * ((3*7)^{-1} mod 5) = 21$
        \item $y_3 = (3 * 5) * ((3 * 5)^{-1} mod 7) = 15$
        \item $x = 2 * y_1 + 3 * y_2 + 4 * y_3$ mod 105
        \item $x \equiv 263 \equiv 53$ mod 105
    \end{enumerate}
\end{frame}


\begin{frame}

    \frametitle{1. Count and Prove}
    \begin{enumerate}[<+->]
        \item Prove $n^{80} \equiv 1$ mod 935, then 5, 11, 17 $\nmid n$.
        \item $n^{80} \equiv 1$ means $n^{80} = 935k + 1$ for some $k$.
        \item $n^{80} \equiv 1$ mod 5
        \item Proof by contradiction:
        \item Assume $5 \mid n$, then $n = 5j$, which means $n \equiv 0$ (mod 5).
        \item Which means $n^{80} \equiv (5j)^{80} \equiv 0^{80} \equiv 0$ mod 5.
        \item Repeat for 11, 17.
    
    \end{enumerate}
\end{frame}

\begin{frame}

    \frametitle{2}
    \begin{enumerate}[<+->]
        \item $p(x) - q(x) = 0$. How many solutions to this?
        \item How many degrees to $p - q$? At most $max(\Delta_1, \Delta_2)$
        \item $f(x) = (x - c)^2$, then $a = -2c$ and $b = c^2$, then $a^2 = 4c^2 = 4b$. 
        \item If $f$ is even, 0. If $f$ is odd, 1.
    \end{enumerate}
\end{frame}

\begin{frame}

    \frametitle{3}
    \begin{enumerate}[<+->]
        \item If $f$ is even, 0. If $f$ is odd, could be 0.
    \end{enumerate}
\end{frame}


\begin{frame}

    \frametitle{3}
    \begin{enumerate}[<+->]
        \item $P(x) = 4 * \Delta_1(x) + 3 * \Delta_2(x) + 0 * \Delta_5(x)$ mod 7
        \item $\Delta_1(x) = \frac{(x - 2)(x-5)}{(1 - 2)(1-5)}$ mod 7
        \item $\Delta_1(x) = \frac{(x - 2)(x-5)}{4}$. Division in a modspace?
        \item $4^{-1} \equiv 2$ mod 7.
        \item $\Delta_1(x) = 2 * (x - 2)(x-5)$.
        \item $P(x) \equiv 4(2x^2 + 6) + 3(2x^2 +2x+3) \equiv 14x^2 + 6x + 33 \equiv 6x + 5$ mod 7
    \end{enumerate}
\end{frame}
\end{document}
